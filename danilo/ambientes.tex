\section{Ambientes}
\frame{
Para compor textos com algum propósito especial, o \LaTeX define
muitos tipos de ambientes para todas as classes de designs.
Em geral, um ambiente é iniciado com o comando $\backslash begin...$ e encerrado com
um $\backslash end....$ Tudo que está entre esses dois comandos é afetado pelo
ambiente.
}

\begin{frame}[fragile]
O ambiente {\bf center} permite que um texto seja centralizado na página;
{\bf flushleft} ajusta o texto à esquerda da página e {\bf ushright} coloca o texto
direita da pàgina. Um exemplo de aplicação são os comandos:
\begin{verbatim}
\begin{center}
Este texto ficará centralizado.
\end{center}

\begin{flushleft}
Este texto ficará à esquerda.
\end{flushleft}

\begin{flushright}
Este texto ficará à direita.
\end{flushright}
\end{verbatim}
\end{frame}


\begin{frame}
\begin{block}{Resultado dos comandos anteriores}
\end{block}

\begin{center}
Este texto ficará centralizado.
\end{center}

\begin{flushleft}
Este texto ficará à esquerda.
\end{flushleft}

\begin{flushright}
Este texto ficará à direita.
\end{flushright}
\end{frame}


\begin{frame}[fragile]
O \LaTeX fornece três ambientes para a criação de listas ({\bf itemize, enumerate e description})

Exemplo de aplicação {\bf itemize:}

\begin{verbatim}
\begin{itemize}
\item itemize
\item enumerate
\item description
\end{itemize}
\end{verbatim}
\begin{itemize}
\item itemize
\item enumerate
\item description
\end{itemize}
\end{frame}


\begin{frame}[fragile]
Exemplo de aplicação {\bf enumerate e description:}.\\
No ambiente description os itens citados não são numerados, 
mas se utilizar um número ou uma letra entre colchetes, este será visualizado em negrito.
\vspace*{-0.5cm}
\begin{columns}
\column[t]{5cm}
\begin{verbatim}
\begin{enumerate}
\item itemize
\item enumerate
\item description
\end{enumerate}
\end{verbatim}
\begin{enumerate}
\item itemize
\item enumerate
\item description
\end{enumerate}

\column[t]{6cm}
\begin{verbatim}
\begin{description}
\item[a)] itemize
\item[b)] enumerate
\item[c)] description
\end{description}
\end{verbatim}
\begin{description}
\item[a)] itemize
\item[b)] enumerate
\item[c)] description
\end{description}
\end{columns}
\end{frame}
