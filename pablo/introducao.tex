\section{Introdução}

\subsection{Projeto RLEG}
\begin{frame}
\begin{block}{Projeto RLEG}
\begin{itemize}
 \item Este trabalho será desenvolvido em parceria com o Laboratório de Automação e Robótica (LARA) da Universidade de Brasília (UnB), 
pois está inserido no âmbito do projeto RLEG, que tem por o objetivo principal a construção de uma prótese
de membro inferior para amputados transfemurais.
\item A primeira fase do projeto foi desenvolvida entre 2005 e 2010, esta etapa não foi testada em humanos.
\end{itemize}
\end{block}
\end{frame}

\frame{
\begin{figure}[ht]
 \centering
 \includegraphics[scale=0.35]{figuras/lara.eps}
\caption{Projeto RLEG fase 1 -- protótipos desenvolvidos no LARA: a) 2005, b)2007, c)2008 e d)2010, fonte: (DIAZ, 2014)}
 \label{nxr}
\end{figure}
}

\frame{
\begin{block}{}
Uma nova versão do projeto RLEG teve início em 2011, com o apoio financeiro da Financiadora de Estudos e Projetos - FINEP.
O projeto aprovado contempla a concepção de uma prótese transfemoral ativa que será testada em pacientes amputados.
No início de 2013 o novo protótipo mecânico da prótese foi concluído.
\end{block}
}


\frame{
\begin{figure}[ht]
 \centering
 \includegraphics[scale=0.22]{figuras/lara2.eps}
\caption{Projeto RLEG fase 2 -- protótipo desenvolvido em 2013,fonte: (DIAZ, 2014)}
 \label{nxr}
\end{figure}
}


\subsection{Justificativa}
\frame{
\begin{block}{Justificativa}
\begin{itemize}
 \item As principais causas de amputação em todo o mundo são as doenças arterial periféricas e as diabetes.
 \item No ano de 2005, um em cada 190 americanos está vivendo com a perda de um membro
e esse número pode dobrar até o ano de 2050, (ZIEGLER-GRAHAM et al., 2008).
 \item No município do Rio de Janeiro entre 1990 e 2000 houve um aumento de cinco vezes 
 na frequência dessas amputações, (SPICHLER et al., 2004).
\end{itemize}
\end{block}
}

\section{Problema e Hipótese}

\begin{frame}
 \begin{block}{Problema}
 \begin{itemize}
 \item  O projeto mecânico da articulação do tornozelo necessita ser executado e o protótipo do pé nacionalizado.    
\end{itemize}

\end{block}

\begin{block}{Hipótese}
\begin{itemize}
 \item Uma prótese ativa da articulação do tornozelo tende a diminuir a ocorrência do choque do pé com o solo,
tendendo a facilitar o uso da prótese transfemoral.
\end{itemize}

\end{block}
\end{frame}


\section{Objetivos}
\frame{
\begin{block}{Objetivo Geral:} 
\begin{itemize}
 \item Desenvolver de um protótipo de prótese do complexo articular tornozelo pé.
\end{itemize}
\end{block}

\begin{block}{Objetivos Específicos:}
\begin{itemize}
\item Caracterizar a condição da passada do membro inferior de um indivíduo não amputado.
\item Modelar a caracterização utilizando o método \textit{Bond Graph}.
\item Revisar e executar o projeto do protótipo mecânico da articulação do tornozelo.
\item Modelar e desenvolver um protótipo de prótese do pé.
\end{itemize}
\end{block}
}

\section{Método}
\subsection{Participantes}
\frame{
\begin{block}{}
 Será realizada pesquisa experimental laboratorial, utilizando o método quantitativo para inferência dos resultados obtidos.
\end{block}

\begin{block}{Participantes}
\begin{itemize}
 \item O protótipo de prótese transfemoral ativa será testado em participantes voluntários amputados com condições clínicas
semelhantes. A escolha dos participantes será realizada pela equipe do projeto RLEG sob coordenação direta do LARA. 
\end{itemize}
\end{block}
}

\subsection{Instrumentos}
\frame{
\begin{block}{Instrumentos}
\begin{itemize}
\item Será utilizado o Laboratório de Automação e Robótica - LARA para coordenação e logística do projeto RLEG.

\item Para a confecção dos dispositivos mecânicos, será utilizado o laboratório de processo de fabricação da FGA e as
instalações do Centro de Formação Profissional Roservarte Alves de Souza (CPF-RAS) - SENAI do Gama, caso necessário.
\end{itemize}

\end{block}

}

\section{Referências}
\begin{frame}
\footnotesize
{[1] -- DIAZ, C. P. O. Caracterização e modelagem de marcha bípede para controle de prótese
transfemural com atuaçãoo magneto-reológica. 56 p. Qualificação Doutorado — Faculdade de Tecnologia, Brasília, 2014.

[2] -- JÚNIOR,C. A. C. et al. Estudo e Desenvolvimento de uma Prótese Ativa de Perna Comandada por Sinais Eletromiográficos. 
Simpósio Brasileiro de Automação Inteligente - SBAI, São Luiz, set. 2005.
 
[3] -- SPICHLER, D. et al. Amputações maiores de membros inferiores por doença arterial
periférica e diabetes melito no município do Rio de Janeiro. J Vasc Br, Sociedade
Brasileira de Angiologia e Cirurgia Vascular., v. 3, n. 2, p. 111–122., 2004. 
 
[4] -- ZIEGLER-GRAHAM, K. et al. Estimating the Prevalence of Limb Loss in the United
States: 2005 to 2050. Arch Phys Med Rehabil, EUA, v. 89, p. 422–429., mar. 2008.
}
 
\end{frame}

\frame{
\begin{center}
  \Huge \textbf {PERGUNTAS ?}
\end{center}
}

\frame{
\begin{center}
  \Huge \textbf {OBRIGADO PELA ATENÇÃO !}
\end{center}
}
















