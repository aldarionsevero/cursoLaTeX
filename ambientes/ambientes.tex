\section{Ambientes}
\begin{frame}[fragile]

Para compor textos com algum propósito especial, o \LaTeX~define
muitos tipos de ambientes para todas as classes de {\it designs}.
Em geral, um ambiente é iniciado com o comando {\code\textbackslash begin\{...\}} e encerrado com
um {\code\textbackslash end\{...\}} Tudo que está entre esses dois comandos é afetado pelo
ambiente.

\vspace{0.5cm}
\begin{center}
\begin{verbatim}
    \begin{AMBIENTE}
        ...
    \end{AMBIENTE}
\end{verbatim}
\end{center}

\end{frame}

\subsection*{Exemplos de Ambientes: Posicionamento de texto} % (fold)

\begin{frame}[fragile]
O ambiente {\bf center} permite que um texto seja centralizado na página;
{\bf flushleft} ajusta o texto à esquerda da página e {\bf ushright} coloca o texto
direita da pàgina. Os seguintes comandos são um exemplo de aplicação:

\vspace{0.5cm}
\begin{CenteredBox}
\begin{lstlisting}[linewidth=6.5cm]
\begin{center}
    Este texto ficará centralizado.
\end{center}

\begin{flushleft}
    Este texto ficará à esquerda.
\end{flushleft}

\begin{flushright}
    Este texto ficará à direita.
\end{flushright}
\end{lstlisting}
\end{CenteredBox}

\end{frame}


\begin{frame}
\begin{block}{Resultado dos comandos anteriores}
\end{block}

\begin{center}
Este texto ficará centralizado.
\end{center}

\begin{flushleft}
Este texto ficará à esquerda.
\end{flushleft}

\begin{flushright}
Este texto ficará à direita.
\end{flushright}
\end{frame}

\subsection*{Exemplos de Ambientes: Listas} % (fold)

\subsubsection*{Itemize} % (fold)
\label{ssub:itemize}

% subsubsection itemize (end)
\begin{frame}[fragile]
O \LaTeX~fornece três ambientes para a criação de listas ({\bf itemize, enumerate e description})

% Exemplo de aplicação {\bf itemize:}

\begin{columns}
\column[t]{6cm}
\begin{verbatim}
    \begin{itemize}
        \item itemize
        \item enumerate
        \item description
    \end{itemize}
\end{verbatim}
\begin{itemize}
    \item itemize
    \item enumerate
    \item description
\end{itemize}
\column[t]{6cm}
\begin{verbatim}
    \begin{enumerate}
        \item itemize
        \item enumerate
        \item description
    \end{enumerate}
\end{verbatim}
\begin{enumerate}
\item itemize
\item enumerate
\item description
\end{enumerate}

\end{columns}
\end{frame}


\subsubsection*{enumerate e description} % (fold)
\begin{frame}[fragile]
No ambiente {\bf description} os itens citados não são numerados, 
mas se utilizar um número ou uma letra entre colchetes, este será visualizado em evidenciado.

\begin{columns}
\column[t]{6cm}
\begin{verbatim}
    \begin{description}
        \item[a)] itemize
        \item[b)] enumerate
        \item[c)] description
    \end{description}
\end{verbatim}
\column[t]{6cm}
\begin{description}
\item[a)] itemize
\item[b)] enumerate
\item[c)] description
\end{description}
\end{columns}
\end{frame}


\subsection*{Exemplos de Ambientes: Matemáticos} % (fold)

\begin{frame}[fragile]
    Um dos ambientes matemáticos mais comuns é o {\code equation}.

\vspace{0.5cm}
\begin{CenteredBox}
\begin{lstlisting}
    \begin{equation}\label{eq1}
        \int_0^{15} x~dx
    \end{equation}
\end{lstlisting}
\vspace{0.5cm}
\end{CenteredBox}
    \begin{equation}\label{eq1}
        \int_0^{15} x~dx
    \end{equation}
\end{frame}

\begin{frame}[fragile]
    Para uma sequencia de equações, é possível usar o {\code eqnarray}.

\vspace{0.5cm}
\begin{CenteredBox}
\begin{lstlisting}
    \begin{eqnarray}\label{eq2}
        3x+y=2\\
        y=2-3x \nonumber \\
        y=2-3.(1)
    \end{eqnarray}\end{lstlisting}
\end{CenteredBox}

    \begin{eqnarray}\label{eq2}
        3x+y=2\\
        y=2-3x \nonumber \\
        y=2-3.(1)
    \end{eqnarray}
\end{frame}

\begin{frame}[fragile]
    Para uma sequencia de equações \textbf{alinhadas}, é possível usar o {\code align}.

\vspace{0.5cm}
\begin{CenteredBox}
\begin{lstlisting}
    \begin{align}\label{eq3}
        2x - 5y &=  8 \\ 
        3x + 9y &=  -12 - 25x + z
    \end{align}
\end{lstlisting}
\end{CenteredBox}

    \begin{align}\label{eq3}
        2x - 5y &=  8 \\ 
        3x + 9y &=  -12 - 25x + z
    \end{align}
\hrulefill\\

\vspace{0.5cm}
\begin{CenteredBox}
\begin{lstlisting}[linewidth=6.5cm]
\begin{align*}\label{eq4}
    w &=z              &  a&=b+c\\
    3w&=\frac{1}{2}z   &  a&=b\\
\end{align*}
\end{lstlisting}
\end{CenteredBox}

    % \begin{align*}\label{eq4}
    %     w &=z              &  a&=b+c\\
    %     3w&=\frac{1}{2}z   &  a&=b
    % \end{align*}

\end{frame}


\begin{frame}[fragile]
    Mas e quando queremos uma única equação com múltiplas linhas? Então usamos o {\code equation} com o {\code split}.

\vspace{0.5cm}
\begin{CenteredBox}
\begin{lstlisting}
  \begin{equation} \label{eq5}
    \begin{split}
        A & = \frac{\pi r^2}{2} \\
         & = \frac{1}{2} \pi r^2
    \end{split}
  \end{equation}
\end{lstlisting}
\end{CenteredBox}    

\begin{equation} \label{eq5}
\begin{split}
    A & = \frac{\pi r^2}{2} \\
     & = \frac{1}{2} \pi r^2
\end{split}
\end{equation}


\end{frame}

\begin{frame}[fragile]
    Todavia, quando é de interesse escrever uma equação simples em meio a um longo texto, é mais simples fazer de uso do ambiente matemático com o uso do caractere \$. 
\vspace{0.3cm}
\begin{center}
\begin{verbatim}
Texto $\oiintclockwise F(\alpha(t)),\alpha’(t)dt$ texto 
    muito longo.
  $$\lim_{h \to 0} \dfrac{f(a+h)-f(a)}{h} =: f’(a)$$
\end{verbatim}
\hrulefill\\
\vspace{0.3cm}
    Texto $\oiintclockwise F(\alpha(t)),\alpha'(t)dt$ texto  muito longo\footnote{Incluir o pacote \textit{pxfonts} para $\oiintclockwise$ funcionar.}.
\vspace{0.25cm}
    $$\lim_{h \to 0} \dfrac{f(a+h)-f(a)}{h} =: f'(a)$$
\end{center}
\end{frame}

\begin{frame}[fragile]
O pacote amsmath também fornece ambientes para construir matrizes e vetores. São eles pmatrix, bmatrix, Bmatrix, vmatrix e Vmatrix, que diferem apenas em qual delimitador é usado ($(\text{ }), [\text{ }], \{\text{ }\}, |\text{ }|$ e $\|\text{ }\|$, respectivamente).
\begin{lstlisting}[linewidth=9cm]
\begin{equation}
A=
\begin{bmatrix}
    a_{11} & a_{12} & a_{13} & \dots  & a_{1n} \\
    a_{21} & a_{22} & a_{23} & \dots  & a_{2n} \\
    \vdots & \vdots & \vdots & \ddots & \vdots \\
    a_{m1} & a_{m2} & a_{m3} & \dots  & a_{mn}
\end{bmatrix}
\end{equation}
\end{lstlisting}
\end{frame}

\begin{frame}[fragile]
\begin{lstlisting}
\begin{equation}
\mathbf{U}(x,y,0)=\mathbf{\bar{U}}(x,y)+0.05
\begin{bmatrix}
\alpha \\
\beta \\
\xi \\
\frac {\mu}{\gamma -1}
\end {bmatrix}
\end{equation}
\end{lstlisting}
\end{frame}
