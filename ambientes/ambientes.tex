\section{Ambientes}
\frame{
Para compor textos com algum propósito especial, o \LaTeX define
muitos tipos de ambientes para todas as classes de designs.
Em geral, um ambiente é iniciado com o comando {\code\textbackslash begin\{...\}} e encerrado com
um {\code\textbackslash end\{...\}} Tudo que está entre esses dois comandos é afetado pelo
ambiente.
}

\subsection*{Exemplos de Ambientes: Posicionamento de texto} % (fold)

\begin{frame}[fragile]
O ambiente {\bf center} permite que um texto seja centralizado na página;
{\bf flushleft} ajusta o texto à esquerda da página e {\bf ushright} coloca o texto
direita da pàgina. Um exemplo de aplicação são os comandos:
\begin{verbatim}
\begin{center}
Este texto ficará centralizado.
\end{center}

\begin{flushleft}
Este texto ficará à esquerda.
\end{flushleft}

\begin{flushright}
Este texto ficará à direita.
\end{flushright}
\end{verbatim}
\end{frame}


\begin{frame}
\begin{block}{Resultado dos comandos anteriores}
\end{block}

\begin{center}
Este texto ficará centralizado.
\end{center}

\begin{flushleft}
Este texto ficará à esquerda.
\end{flushleft}

\begin{flushright}
Este texto ficará à direita.
\end{flushright}
\end{frame}

\subsection*{Exemplos de Ambientes: Listas} % (fold)

\begin{frame}[fragile]
O \LaTeX fornece três ambientes para a criação de listas ({\bf itemize, enumerate e description})

Exemplo de aplicação {\bf itemize:}

\begin{columns}
\column[t]{6cm}
\begin{verbatim}
    \begin{itemize}
        \item itemize
        \item enumerate
        \item description
    \end{itemize}
\end{verbatim}
\column[t]{6cm}
\vspace{0.5cm}
\begin{itemize}
    \item itemize
    \item enumerate
    \item description
\end{itemize}
\end{columns}
\end{frame}



\begin{frame}[fragile]
Exemplo de aplicação {\bf enumerate e description:}.\\
No ambiente {\bf description} os itens citados não são numerados, 
mas se utilizar um número ou uma letra entre colchetes, este será visualizado em evidenciado.
\vspace*{-0.5cm}
\begin{columns}
\column[t]{5cm}
\begin{verbatim}
    \begin{enumerate}
        \item itemize
        \item enumerate
        \item description
    \end{enumerate}
\end{verbatim}
\begin{enumerate}
\item itemize
\item enumerate
\item description
\end{enumerate}

\column[t]{6cm}
\begin{verbatim}
    \begin{description}
        \item[a)] itemize
        \item[b)] enumerate
        \item[c)] description
    \end{description}
\end{verbatim}
\begin{description}
\item[a)] itemize
\item[b)] enumerate
\item[c)] description
\end{description}
\end{columns}
\end{frame}


\subsection*{Exemplos de Ambientes: Equações} % (fold)

\begin{frame}[fragile]
    Um dos ambientes matemáticos mais comuns é o {\code equation}.

    \begin{equation}\label{eq1}
        \int_0^{15} x~dx
    \end{equation}
\end{frame}

\begin{frame}[fragile]
    Para uma sequencia de equações, é possível usar o {\code eqnarray}.

    \begin{eqnarray}\label{eq2}
        3x+y=2\\
        y=2-3x \nonumber \\
        y=2-3.(1)
    \end{eqnarray}
\end{frame}

\begin{frame}[fragile]
    Para uma sequencia de equações \textbf{alinhadas}, é possível usar o {\code align}.

    \begin{align}\label{eq3}
        2x - 5y &=  8 \\ 
        3x + 9y &=  -12 - 25x + z
    \end{align}
\hrulefill\\
    \begin{align*}\label{eq4}
        x&=y           &  w &=z              &  a&=b+c\\
        2x&=-y         &  3w&=\frac{1}{2}z   &  a&=b\\
        -4 + 5x&=2+y   &  w+2&=-1+w          &  ab&=cb
    \end{align*}

\end{frame}


\begin{frame}[fragile]
    Mas e quando queremos uma única equação com múltiplas linhas? Então usamos o {\code equation} com o {\code split}.

    

\begin{equation} \label{eq5}
\begin{split}
    A & = \frac{\pi r^2}{2} \\
     & = \frac{1}{2} \pi r^2
\end{split}
\end{equation}


\end{frame}

\begin{frame}[fragile]
    Todavia, quando é de interesse escrever uma equação simples em meio a um longo texto, é mais simples fazer de uso do ambiente matemático com o uso do caractere \$. 
\vspace{0.5cm}
\begin{center}
    
\begin{verbatim}
Texto $\oiintclockwise F(\alpha(t)),\alpha’(t)dt$ texto 
    muito longo.
  $$\lim_{h \to 0} \dfrac{f(a+h)-f(a)}{h} =: f’(a)$$
\end{verbatim}
\hrulefill\\
\vspace{0.5cm}
    Texto $\oiintclockwise F(\alpha(t)),\alpha’(t)dt$ texto  muito longo\footnote{Incluir o pacote \textit{pxfonts}}.
\vspace{0.25cm}
    $$\lim_{h \to 0} \dfrac{f(a+h)-f(a)}{h} =: f’(a)$$
\end{center}

\end{frame}