\section[Inclusão de códigos externos em \LaTeX]{Inclusão de códigos externos em \LaTeX~(listings)}

\begin{frame}[fragile]
\begin{block}{}
Para incluir códigos externos em um documento \LaTeX~usamos o pacote {\it listings}. Com ele podemos:
\begin{itemize}
 \item Adicionar texto não formatado (similar ao verbatim) mas com foco em código fonte.
 \item Indicar a linguagem a ser mostrada, visto que, dependendo da linguagem a formatação muda.
 \item Incluir somente trechos do código fonte.
 \item Incluir arquivos de código externo na integra.
\end{itemize}
\end{block}
\end{frame}


\begin{frame}[fragile]

Para incluir código fonte sem arquivo externo basta usar o comando: 

\begin{verbatim}
\begin{lstlisting} 
    Put your code here. 
\end{lstlisting}
\end{verbatim}


\end{frame}


\begin{frame}[fragile]

Para incluir código fonte com um arquivo externo basta usar o comando: 

\begin{verbatim}
 
\lstinputlisting{source_filename.py}

\end{verbatim}


\end{frame}

\begin{frame}[fragile]

Para incluir código fonte com um arquivo externo, definir a linguagem e escolher as linhas que serão mostradas basta usar o comando: 

\begin{verbatim} 
\lstinputlisting[language=Python, 
firstline=37, lastline=45]{source_filename.py}

\end{verbatim}

Isso possibilita manter um documento \LaTeX~estável, mesmo que dependa de um código fonte externo que esteja sofrendo atualizações constantemente. Assim, a cada mudança do código, o documento é atualizado de imediato sem a necessidade de atualizar o arquivo {\ttfamily tex}.


\end{frame}


%\begin{frame}[fragile]
%
%Para incluir código fonte com um arquivo externo basta usar o comando: 
%
%\begin{description}[maiortextodomundoqueconsigoea]
%\item [\textbackslash lstinputlisting\{source_filename.py\}]
%\end{description}
%
%\end{frame}